\section{Conclusion} \label{sec:conclusion}
Machine learning is a difficult field, with unlimited of options in the hyper parameters, activation functions and so on. This conclusion is therefore just based on the things I have done, being aware that things probably can be done in a better way.

For the energy recognition, both linear regression and neural networks give great results, which we would expect since the energy is linear with respect to the spin configurations. What is more surprising is how good accuracy we are able to get for the classification problem, in particular when using neural networks. We obtained an accuracy above 99\% for the test cause with only one minimization iteration, with CPU-time of just a few seconds in a high-level language.

Even though the results are satisfying in this project, we are not 100\% satisfied before we can guarantee obtaining the correct phase. I believe that other activation functions together with more complex optimization techniques can increase the test accuracy to 100\%. If I had time, I would have like to implement a momentum method or a conjugate gradient method to see if we were able to reach a lower minimum. 