\section{Methods} \label{sec:methods}

\subsection{Resampling techniques} \label{sec:resampling}
A resampling technique is a way of estimating the variance of data sets without calculating the covariance. As we saw in section section \ref{sec:error_analysis}, the true covariance is given by a double loop which we will avoid calculating if possible. There are different ways of doing this, and we have already went through several of them in this course:
\begin{itemize}
\item{Jackknife resampling}
\item{K-fold validation}
\item{Bootstrap method}
\item{Blocking method}.
\end{itemize}

For this particular project we have been focusing on the bootstrap and the K-fold validation methods, so only they will be covered here.

\subsubsection{Bootstrap method} \label{sec:bootstrap}
When we construct a data set, we usually draw samples from a probability density function (PDF) and get a set of samples $\vec{x}$. If we draw a large number of samples, the sample variance will approach the variance of the PDF. The bootstrap method turns this upside down, and tries to estimate the PDF given a set data set, because if we know the PDF, we know in principle everything about the data set. 

The assumption we need to make, is that the relative frequency of $x_i$ equals $p(x_i)$, which is reasonable (for instance, think about how the histogram looks like when we draw samples from a normal distribution). In this project the vector that we want to find, $\vec{\beta}(x,y)$, is a function of two set of variables. Fortunetely they are independent of eachother, so we can safely apply the independent bootstrap on them separately. The independent bootstraps goes as
\begin{enumerate}
\item Draw $n$ samplings from the data set $\vec{x}$ with replacement and denote the new data set as $\vec{x}^*=\{x_1,x_2,\hdots,x_n\}$
\item Compute $\vec{\beta}(\vec{x}^*)\equiv\vec{\beta}^*$
\item Repeat the procedure above $K$ times
\item The average value of all $K$ $\vec{\beta}^*$'s are stored in a new vector $\vec{\bar{\beta}}^*$
\item Finally, the variance of $\vec{\bar{\beta}}^*$ should correspond to the sample variance
\end{enumerate}
\cite{BootstrapEfron}
[Efron, B. Bootstrap Methods: Another Look at the Jackknife. Ann. Statist. 7 (1979), no. 1, 1--26. doi:10.1214/aos/1176344552. ]

The implementation could look something like this
\lstset{basicstyle=\scriptsize}
\begin{lstlisting}
def bootstrap(data, K=1000):
    dataVec = np.zeros(K)
    for k in range(K):
        dataVec[k] = np.average(np.random.choice(data, len(data)))
    Avg = np.average(dataVec)
    Var = np.var(dataVec)
    Std = np.std(dataVec)
    
    return Avg, Var, Std
\end{lstlisting}

\subsubsection{K-fold validation method} \label{sec:kfold}


\subsection{Minimization methods} \label{sec:minimization}
When the interaction term is excluded, we know which $\alpha$ that corresponds to the energy minimum, and it is in principle no need to try different $\alpha$'s. However, sometimes we have no idea where to search for the minimum point, and we need to try various $\alpha$ values to determine the lowest energy. If we do not know where to start searching, this can be a time consuming activity. Would it not be nice if the program could do this for us?

In fact there are multiple techniques for doing this, where the most complicated ones obviously also are the best. Anyway, in this project we will have good initial guesses, and are therefore not in need for the most fancy algorithms. 

\subsubsection{Gradient Descent} \label{sec:gd}
Perhaps the simplest and most intuitive method for finding the minimum is the gradient descent method (GD), which reads
\begin{equation}
\label{eq:GD}
\beta_i^{\text{new}}=\beta_i - \eta\cdot\frac{\partial Q(\beta_i)}{\partial\beta_i}
\end{equation}
where $\beta_i^{\text{new}}$ is the updated $\beta$ and $\eta$ is a step size, in machine learning often refered to as the learning rate. The idea is to find the gradient of the cost function $Q(\vec{\beta})$ with respect to a certain $\beta_i$, and move in the direction which minimizes the cost function. This is repeated until a minimum is found, defined by either
\begin{equation}
\frac{\partial Q(\beta_i)}{\partial\beta_i}<\varepsilon
\end{equation}
or that the change in $\beta_i$ for the past $x$ steps is small. 
\par 
\vspace{3mm}

Before we can implement equation \eqref{eq:GD}, we need an expression for the derivative of $Q$ with respect to $\beta_i$. The general form of the cost function as discussed in section \ref{sec:general} reads
\begin{equation}
Q(\vec{\beta},\lambda,q)=\sum_{i=1}^{n}\Big(y_i-\beta_0-\sum_{j=1}^px_{ij}\beta_j\Big)^2+\lambda\sum_{j=1}^p\beta_j^q,
\label{eq:cost_gen}
\end{equation}
and its derivative with respect to $\beta_k$ is
\begin{equation}
\frac{\partial Q(\vec{\beta},\lambda,q)}{\partial\beta_k}=-2\sum_i^n\Big(y_i-\beta_0-\sum_{j=1}^px_{ij}\beta_j\Big)x_{ik}+q\lambda\beta_k^{q-1}.
\label{eq:der_cost_gen}
\end{equation}
The vectorized version looks like
\begin{equation}
\frac{\partial Q(\vec{\beta},\lambda,q)}{\partial\vec{\beta}}=-2\hat{X}^T(\vec{y}-\hat{X}\vec{\beta})+q\lambda\vec{\beta}^{q-1}
\label{eq:der_cost_gen_vec}
\end{equation}

The algorithm of this minimization method is thus as follows:

\lstset{basicstyle=\scriptsize}
\begin{lstlisting}

while dbeta > epsilon:
	e = y - X.dot(beta)
    debeta = 2*X.T.dot(e) - q*\lambda*np.power(abs(beta), q-1)
    beta += \eta*dbeta


\end{lstlisting}