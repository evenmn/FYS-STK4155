\section{Theory} \label{sec:theory}

\subsection{Sound analysis}

\subsubsection{Time domain}
Sounds are longitudinal waves which actually are different pressures in the air. They are usually represented by functions giving pressure per time, where a pure tone has a repeating function. This function is obviously continuous, but since computers discretize functions, we will loose some information, see Figure ...

INSERT FIGURE

How much information we loose depends on the sampling frequency (or sampling rate), which is the number of sampling points per second. A rule of thumb is that one should have twice as high sampling rate as the highest sound frequency to keep the most important information. For instance, a human ear can perceive frequencies in the range 20-20000Hz, so around a sampling rate around 40kHz is known to be satisfying. Ordinary CD's use a sampling rate of 44.1kHz. 

\subsubsection{Frequency domain}
Sometimes, a frequency domain rather than the time domain gives a better picture. On one hand, one looses the time dependency, but on the other one gets information about which frequencies that are found in the wave. One goes from the time domain to the frequency domain using Fourier transformations. Which one that gives the best results in our case is not clear. 

INSERT FIGURE

\subsection{Urban Sound Challenge}
The Urban Sound Challenge is a classification challenge provided by Analytics Vidhya with the purpose of introducing curious people to a real-world classification problem. To receive the data set, one needs to register for the challenge and is then expected to summit the solutions in 31st of Desember 2018. For more practical information, see [\url{https://datahack.analyticsvidhya.com/contest/practice-problem-urban-sound-classification/}]

The data set consists of 8732 sound samples with a constant sampling rate of 22050Hz and the lengths are maximum 4 seconds, but vary. This makes the training session difficult, since we will get input arrays of different lengths. The samplings are distributed between ten classes, namely
\begin{multicols}{2}
\begin{itemize}
	\setlength\itemsep{0.2em}
	\item air conditioner
	\item car horn
	\item children playing
	\item dog bark
	\item drilling
\end{itemize}

\columnbreak

\begin{itemize}
	\setlength\itemsep{0.2em}
	\item engine idling
	\item gun shot
	\item jackhammer
	\item siren
	\item street music
\end{itemize}
\end{multicols}

ADD SOME EXAMPLES